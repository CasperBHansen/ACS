%==============================================================================%
% DISCUSSION                                                                   %
%==============================================================================%

\newpage
\section{Discussion on Concurrent Implementation}

\subsection{Description of implementation}
Our implementation maintains a table of locks which binds an ISBN to an
associated lock. In addition to this, we maintain a global lock, which acts as
a safeguard on the lock table. Its purpose is to ensure that we only lock whole
sets of books at a time. Obtaining write locks concurrently is therefore possible
if and only if the sets of books do not intersect. This enforces conflict
resolution.

\subsection{Locking protocol correctness}
We reason that our locking protocol holds, because we lock sets of books, rather
than acquiring individual locks as they are needed. Our implementation ensures
that we cannot obtain a subset of locks for a given operation, without stalling
if one or more of the locks needed cannot be acquired, if it is already in use.
Hence any operation with intersecting sets of books must wait for the previous
operation to finish before the next operation can proceed.

\subsection{Deadlocks}
...

\subsection{Scalability and bottlenecks}
...

\subsection{Overhead}
...
