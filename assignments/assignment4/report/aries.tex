%==============================================================================%
% ARIES                                                                        %
%==============================================================================%

\section{ARIES}

The ARIES recovery method starts at the most recent checkpoint. Since ARIES
use fuzzy checkpoints, only the transaction table ({\tt TT}) and dirty page
table ({\tt DPT}) is given, and these are empty in this case.

\noindent
The three phases proceeds as follows

\noindent
{\bf Analysis:}

In this phase the log is read from checkpoint until the most recent entry in
the log.

\begin{multicols}{2}

{\tt LSN 1} is the starting checking. As we look ahead, we find the {\tt TT} and
{\tt DPT} to be empty at {\tt LNS 2}, so we scan the log until the time of the
crash occurs.

\begin{figure}[H]
    \centering
    \begin{tabular}{|l|l|l|l|}
        \hline
        {\bf LSN} & {\bf Action} & {\bf Page ID} & {\bf recLSN} \\ \hline
        3 & Add     & P2 & 3  \\ \hline
        4 & Add     & P1 & 4  \\ \hline
        5 & Add     & P5 & 5  \\ \hline
        6 & Add     & P3 & 6  \\ \hline
    \end{tabular}
    \caption{Dirty page table changes}
    \label{fig:dpt}
\end{figure}

\colbreak

\begin{figure}[H]
    \centering
    \begin{tabular}{|l|l|l|l|l|}
        \hline
        {\bf LSN} & {\bf Action}   & {\bf TID}  & {\bf State}  & {\bf lastLSN} \\ \hline
        3  & Add     & T1 & U & 3 \\ \hline % wrong last
        \rowcolor{awesomecolor}
        4  & Alter   & T1 & U & 4 \\ \hline % also probably wrong last maybe
        5  & Add     & T2 & U & 5 \\ \hline
        6  & Add     & T3 & U & 6 \\ \hline
        7  & Alter   & T3 & C & 7 \\ \hline
        8  & Alter   & T2 & U & 8 \\ \hline
        \rowcolor{awesomecolor}
        9  & Alter   & T2 & U & 9 \\ \hline
        10 & Remove  & T3 & - & - \\ \hline
    \end{tabular}
    \caption{Transaction table changes (remaining entries are colored)}
    \label{fig:tt}
\end{figure}

\end{multicols}

Note: We acknowledge that you say that {\it ``both''} transactions (There's three?) have some wrong {\tt Last\_LSN}, but we have not changed them. In accordance with the textbook, the transaction tables {\tt Last\_LSN} field contains the log LSN of the most recent log entry for the given transaction. This is also enforced in the textbook description of the analysis phases wherein it is stated that the {Last\_LSN} is set to the current LSN when reading the log from checkpoint to time of crash.\\

\iffalse
\begin{enumerate}
\item {\tt LNS: 1} - Start of checkpoint. We here look ahead to find the
{\tt DPT} and {\tt TT}. These are found empty at {\tt LNS: 2}. The log is now
scanned until time of crash.
\item {\tt LNS: 3} - Add {\tt T1} as {\tt U} with LAST\_LSN set to {\tt 3} in {\tt TT} and {\tt P2} to {\tt DPT} with recLSN equal 3.
\item {\tt LNS: 4} - Add {\tt P1} to {\tt DPT} with recLSN equal 4 and change LAST\_LSN of
    {\tt T1} set to 4 in {\tt TT}.
\item {\tt LNS: 5} - Add {\tt T2} at {\tt U} with LAST\_LSN set to {\tt 5}
    to {\tt TT} and {\tt P5} to {\tt DPT} with recLSN equal 5.
\item {\tt LNS: 6} - Add {\tt T3} as {\tt U} with LAST\_LSN set to {\tt 6}
    to {\tt TT} and {\tt P3} to {\tt DPT}  with recLSN equal 6.
\item {\tt LNS: 7} - Change {\tt T3} to {\tt C} and change LAST\_LSN to 7 in {\tt TT}.
\item {\tt LNS: 8} - Change LAST\_LSN of {\tt T2} to 8 in {\tt TT}.
\item {\tt LNS: 9} - Change LAST\_LSN of {\tt T2} to 9 in {\tt TT}.
\item {\tt LNS: 10} - Remove {\tt T3} from {\tt TT}.
\end{enumerate}
\fi

\noindent
{\bf Redo:}

In this phase, the dirty page table is used to determine the earliest LSN that needs to be redone. Since no pages have been updated (un-dirtied) all pages in the log since the checkpoint is in the {\tt DPT}, it can also therefor never happen that recLSN or page LSN is greater than the current LSN being redone. Thus all updates logged at LSNs $3,4,5,6,8,9$ are redone.

\noindent
{\bf Undo:}

The undo reads the transaction table and undo's action of the transactions present (the loser transaction) and adding compensation log entries (CLR) to the log.

\subsection{Transaction state and dirty pages after analysis phase}
Following the above the transaction table includes
\begin{itemize}
\item {\tt T1} with status {\tt U} and LAST\_LSN equal to 3.
\item {\tt T2} with status {\tt U} and LAST\_LSN equal to 8.
\end{itemize}
and the dirty page table holds
\begin{itemize}
\item {\tt P1} with recLSN equal to 4.
\item {\tt P2} with recLSN equal to 3.
\item {\tt P3} with recLSN equal to 6.
\item {\tt P5} with recLSN equal to 5.
\end{itemize}

\subsection{The sets of winners and loser transactions}
The losers are {\tt T1} and {\tt T2}, because they are both in the transaction
table at the time of the analysis phase --- meaning they have to be undone.
Thus the only winner is {\tt T3}.

\subsection{Values for LSNs at redo phase start and undo phase end}
The redo phase starts at LSN 3 as this is the smallest recLSN of the {\tt DPT}. The undo phase ends at LSN 3 again because the {\tt T1} modifying {\tt P2} recLSN is present in the {\tt TT} at the start of the undo phase and must thus be undone.

\subsection{Set of log records that may rewrite during redo phase}
As mentioned in the redo section, these are $\{3,4,5,6,8,9\}$.

\subsection{Set of log records undone}
As mentioned in the undo section, these are $\{3,4,5,8,9\}$.

\subsection{Contents of the log after recovery}
The log after the recovery can be seen in Table \ref{fig:log}.

\begin{figure}[H]
\centering
\begin{tabular}{|l|l|l|l|l|l|}
\hline
LSN & LAST\_LSN & TRAN\_ID & TYPE       & PAGE\_ID & undoNextLSN \\ \hline
1   & -         & -        & BEGIN CKPT & -        & -           \\ \hline
2   & -         & -        & END CKPT   & -        & -           \\ \hline
3   & NULL      & T1       & UPDATE     & P2       & -           \\ \hline
4   & 3         & T1       & UPDATE     & P1       & -           \\ \hline
5   & NULL      & T2       & UPDATE     & P5       & -           \\ \hline
6   & NULL      & T3       & UPDATE     & P3       & -           \\ \hline
7   & 6         & T3       & COMMIT     & -        & -           \\ \hline
8   & 5         & T2       & UPDATE     & P5       & -           \\ \hline
9   & 8         & T2       & UPDATE     & P3       & -           \\ \hline
10  & 7         & T3       & END        & -        & -           \\ \hline
*   & *         & *        & *CRASH*    & *        & *           \\ \hline
11  & 9         & T2       & ABORT      & -        & -           \\ \hline
12  & 4         & T1       & ABORT      & -        & -           \\ \hline
13  & 11        & T2       & CLR        & P3       & 8           \\ \hline
14  & 13        & T2       & CLR        & P5       & 5           \\ \hline
15  & 14        & T2       & CLR        & P5       & NULL        \\ \hline
16  & 15        & T2       & END        & -        & -           \\ \hline
17  & 12        & T1       & CLR        & P1       & 3           \\ \hline
18  & 17        & T1       & CLR        & P2       & NULL        \\ \hline
19  & 18        & T1       & END        & -        & -           \\ \hline
\end{tabular}
\caption{Log}
\label{fig:log}
\end{figure}


