%==============================================================================%
% RECOVERY CONCEPTS                                                            %
%==============================================================================%

\section{Recovery Concepts}

\subsection{Necessity of redo/undo in systems force and no-steal}
Using a force, no-steal approach, if we assume we use stable storage,
every change is forced to disk as soon as a commit is reached. Following a
crash the database is already at a consistent state, it is not needed to
perform either a redo or undo phase. If storage is not stable, a crash may
have occured during writing. (unsure about everything from this point on)
Since we we use {\bf WAL} the commit log entry is written before a commit
takes place, and we may need to redo transactions marked with a {\tt commit} but not
and {\tt end} entry.

\subsection{Differences between non-volatile and stable storage}
Casper:
The main difference of these types of storage is that stable storage relies on
book keeping to restore its data in the event of a system crash, while data in
non-volatile storage aren't affected by crashes.

Hans:
non-volatile as well as stable storage retains stored data in the event of a
system shutdown or crash. The difference is that stable storage ensures
atomicity of operations, and thus system crashes may ruin the state of
non-volatile storage while stable storage is still in consistent state.

\subsection{Write-Ahead logging}

The two situations when the log tail must be force written to stable storage
is when a commit is performed and ... well I don't know.
Wiki (for discussion): "A record of the changes must still be preserved at commit time to ensure that the transaction is durable. This record is typically written to a sequential transaction log, with the actual changes to the database objects being changes which can be written at a later time.

For frequently changed objects, a no-force policy allows updates to be merged and so reduce the number of write operations to the actual database object. A no-force policy also reduces the seek time required for a commit by having mostly sequential write operations to the transaction log, rather than requiring the disk to seek to many distinct database objects during a commit."