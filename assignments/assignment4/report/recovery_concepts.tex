%==============================================================================%
% RECOVERY CONCEPTS                                                            %
%==============================================================================%

\section{Recovery Concepts}

\subsection{Necessity of redo/undo in systems force and no-steal}
Using a force, no-steal approach, every change is forced to disk as soon as a commit is reached. Following a
crash the database is already at a consistent state, it is not needed to
perform either a redo or undo phase.

\subsection{Differences between non-volatile and stable storage}

Non-volatile as well as stable storage retains stored data in the event of a
system shutdown or crash. The difference is that stable storage ensures
atomicity of operations, and thus system crashes may ruin the state of
non-volatile storage while stable storage is still in consistent state.

\subsection{Write-Ahead logging}

The two situations when the log tail must be force written to stable storage
is when a commit is performed and when dirty pages are written to disk.
Write-ahead logging must maintain the invariant that any writing occurs in the
log before it is reflected on disk. As a natural consequence of the nature of
a commit, of course, the log tail must be written to stable storage. Likewise,
dirty pages that are evicted, will get written to disk, and hence requires the
log tail to be force written.

The main function of this strategy is to accomplish durability in the event of
system crashes. Because each transaction is preserved in stable storage, even
when the system crashes, any and all changes made, but not yet reflected on disk
are recoverable. The aforementioned scenarios are the only actions that constitute
changes on disk, and because the log tail holds a backward reference, we are
able to trace back through the changes made and restore any otherwise lost
transactions. The scenarios we want to prevent are, in case of the commit, that a user of the database gets the response that his transaction has committed successfully, but a crash wiped the log entry of this event that may not have been written to non-volatile storage yet, so recovery is incomplete. In case of the update log entry, we need this to log entry to be force written to stable storage to prevent data from being present in the database but the log not reflecting the changes took place.

\iffalse
Wiki (for discussion): "A record of the changes must still be preserved at commit time to ensure that the transaction is durable. This record is typically written to a sequential transaction log, with the actual changes to the database objects being changes which can be written at a later time.

For frequently changed objects, a no-force policy allows updates to be merged and so reduce the number of write operations to the actual database object. A no-force policy also reduces the seek time required for a commit by having mostly sequential write operations to the transaction log, rather than requiring the disk to seek to many distinct database objects during a commit."
\fi
