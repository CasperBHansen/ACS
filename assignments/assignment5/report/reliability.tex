%==============================================================================%
% RELIABILITY                                                                  %
%==============================================================================%

\section{Reliability}

\subsection{Probability of daisy chain connectivity}
A daisy-chained network, assuming a linear topology, the graph consists of
\begin{align}
    c &= n - 1 \enspace ,
\end{align}
connections.

The probability of a link failure is $p$. The probability that a link doesn't
fail is thus the complement $1 - p$. The probability that all nodes are connected
is then equivalent with the case that none of links fail. Under the assumption
that each link failure is independent, we can then simply multiply the individual
probabilities. We then have
\begin{align}
    (1 - p)^{n-1} \enspace .
\end{align}

In our case $n = 3$.

\subsection{Probability of fully-connected network connectivity}
A fully connected network with 3 nodes, there's 3 links that can fail. As long
as 2 or more links are working, the network is connected. We found it easiest
to look at the possible ways to have a connected network and add the
individual probabilities.

There's 3 ways where 1 link can fail with probability $p(1-p)^2$ and 1 way
0 links can fail with probability $(1-p)^3$. The probability that a
fully-connected network is working is then
\begin{align*}
3p(1-p)^2 + (1-p)^3 \enspace .
\end{align*}

\subsection{Hypothetical town network}

by simply computing the 2 probabilities, we get
\begin{align*}
p_{daisy} &= (1-0.000001)^{3-1} \approx 99.9998\enspace \% , \\
p_{full} &= 3 \cdot 0.0001 \cdot (1-0.0001)^2+(1-0.0001)^3 \approx 99.999997 \% \enspace .
\end{align*}
The above shows that the less reliable links in a fully connected network
makes a more reliable overall network than the daisy chained network.

