%==============================================================================%
% RELIABILITY                                                                  %
%==============================================================================%

\section{Reliability}

\subsection{Probability of daisy chain connectivity}
A daisy-chained network, assuming a linear topology, the graph consists of
\begin{align}
    c &= n - 1 \enspace ,
\end{align}
connections.

The probability of a link failure is $p$. The probability that a link doesn't
fail is thus the complement $1 - p$. The probability that all nodes are connected
is then equivalent with the case that none of links fail. Under the assumption
that each link failure is independent, we can then simply multiply the individual
probabilities. We then have

\begin{align}
    (1 - p)^{n-1} \enspace .
\end{align}

\dots

\subsection{Probability of fully connected network connectivity}
A fully connected network consists of
\begin{align}
    c &= \frac{n^2 - n}{2} \enspace ,
\end{align}
connections.

In order for a fully connected network to not connect all buildings, a node
would have to lose $n-1$ connections, which would have a probability of
$q = p^{n-1}$ for each individual node. Again, by the assumption of independence,
as before, we can multiply these
\begin{align}
    q^{n-1}
\end{align}

\subsection{Hypothetical town network}
\dots

