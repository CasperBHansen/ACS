%==============================================================================%
% DISTRIBUTED COORDINATION                                                     %
%==============================================================================%

\section{Distributed Coordination}
The three-phase commit protocol is, unlike the two-phase commit protocol, an
asynchronous protocol, because in the two-phase protocol each yes-voting
participant must wait for the coordinator to reply to the {\it getDecision}
request. Likewise, the coordinator must wait for all votes of the {\it canCommit?}
requests, in order to proceed.

This means that in the event of coordinator or participant failure, all
participants are stalled until an abort mechanism is triggered --- such as a
request time-out.

In the event that a coordinator or other participants fail during a transaction,
the three-phase commit protocol ensures that participants that have voted ``yes''
have entered a state (e.g. lock acquisition) in which they can carry out their
respective transactions independently of individual failures. Once a participant
has entered the second phase it will eventually commit, regardless of failures.

Although not specified in the assignment text, or the textbook, we have learned
that in three-phase commit protocols, unlike the two-phase commit, is asynchronous
and employs timers that belong to each participant and the coordinator. When the
coordinator learns of a failure (e.g. due to time-out) or negative answer of one
or more participants, it broadcasts a {\it doAbort} to all of the participants.
Likewise, in the event that a request on the participant-side times-out, the
coordinator is assumed to experience a failure. This triggers either an abort, if
we are in the first phase, or a commit if we are in the second phase.
