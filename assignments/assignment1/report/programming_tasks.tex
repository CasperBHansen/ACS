%==============================================================================%
% PROGRAMMING TASKS                                                            %
%==============================================================================%

\newpage
\section{Programming Tasks}

\setlength\columnsep{30pt}
\begin{multicols}{2}
    In order to implement the functionality, we added the message tags for the
    desired functions to {\tt BookStoreMessageTags} --- namely
    {\tt GETBOOKSINDEMAND}, {\tt RATEBOOKS} and {\tt GETTOPRATEDBOOKS}, as
    shown in excerpt \ref{code:utils/BookStoreMessageTag.java}.
    
    \colbreak
    
    \codeexcerpt{utils/BookStoreMessageTag.java}{12}{14}
\end{multicols}
\setlength\columnsep{10pt}

\subsection{\tt rateBooks}
The implementation of {\tt rateBooks} is rather straight forward. First we check
that the argument isn't null. Then, for each of the ratings passed to us, we
validate that the ISBN is valid and contained within the book map, and lastly
that the rating is actually valid. If any of these fails, we throw an exception.
If not, we proceed to perform the rating.
\codeexcerpt{business/CertainBookStore.java}{333}{370}

Now that the functionality is implemented, we can add the message handling code.

Firstly, we parse the serialized XML from the received POST request. Then we
attempt to perform the rating using the set of book ratings. If this fails we
catch the generated exception. And lastly, if all went well, we respond
accordingly.
\codeexcerpt{server/BookStoreHTTPMessageHandler.java}{251}{266}

Now the server is able to process book ratings, so all we have to do is to
expose the client-side to this API. To do so, we simply serialize the book
ratings and send the POST request with it and wait for a response from the
server.
\codeexcerpt{client/BookStoreHTTPProxy.java}{132}{144}

\subsection{\tt getTopRatedBooks}

\codeexcerpt{business/CertainBookStore.java}{280}{311}

\subsection{\tt getTopRatedBooks}
\codeexcerpt{business/CertainBookStore.java}{313}{330}
