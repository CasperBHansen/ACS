%==============================================================================%
% DISCUSSION ON ARCHITECTURE                                                   %
%==============================================================================%

\section{Discussion on Architecture}

\subsection{Question 1}
We follow the {\it all-or-nothing} semantics by way of exceptions; whenever an
exception occurs, nothing is changed. All checks occur beforehand, and only when
everything has been validated we perform the requested action.

\subsection{Question 2}

\subsubsection{Sub-Question a}

The architecture is strongly modular because it is based on message tags, which
allows different stores to handle each request differently. There is a very
clear separation between messages and the action being performed, which greatly
increases the overall extensibility and ease of maintenance of the system.
Another powerfull advantage of the modularity in this design is that it also
allows us to implement features that aren't necessarily exposed to the client or
business, before all components work properly, by way of simply not exposing the
message tag. Hence we can test in a live environment, without risking the client
or business to be affected by any anomalies as we do so.

\subsubsection{Sub-Question b}

There's two main design decisions

\subsection{Question 3}

\subsection{Question 4}

In the architecture, this can be seen from the way HTTP communication is handled. Clients send and receive messages to the bookstore using the Jetty 8 html client/server library, wrapped in the {\tt SendAndRecv} method. Since the client does not attempt to resend messages, and since the bookstore servers plan for handling errors is simply to pass the on to the client, the architecture exhibits an {\it ``At most once''} behavior.

\subsection{Question 5}

\subsection{Question 6}

\subsection{Question 7}


