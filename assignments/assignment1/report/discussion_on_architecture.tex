%==============================================================================%
% DISCUSSION ON ARCHITECTURE                                                   %
%==============================================================================%

\section{Discussion on Architecture}

\subsection{Question 1}

\subsubsection{}
We follow the {\it all-or-nothing} semantics by way of exceptions; whenever an
exception occurs, nothing is changed. All checks occur beforehand, and only when
everything has been validated we perform the requested action. With regards to
testing this, our tests are oriented towards failing by producing such an
exception for cases where it can fail.

For a description of the implementation, please refer to the {\it Programming
Tasks} section.

\subsubsection{}
...

\subsection{Question 2}

\subsubsection{}
The architecture is strongly modular due to two main design choices.  One is the use of message tags, which
allows different stores to handle each request differently. There is a very
clear separation between messages and the action being performed, which greatly
increases the overall extensibility and ease of maintenance of the system.
Another powerful advantage of the modularity in this design is that it also
allows us to implement features that aren't necessarily exposed to the client or
business, before all components work properly, by way of simply not exposing the
message tag. Hence we can test in a live environment, without risking the client
or business to be affected by any anomalies as we do so. The second design choice is due to the powerful usage of interfaces in the RPC design. The interfaces together define everything we want the bookstore to do, thus the bookstore implements both interfaces. Each interface then define the two different ways to interact with the bookstore. We recognize how the server is central to the HTTP RPC design but stress that since the proxies at the end of the message pathways as well implement the interface, the RPC architecture is hidden from users of the bookstore creating an illusion that they interface directly with the bookstore.

\subsubsection{}
There is a clear separation between the client and the service, as neither
communicates directly with each other. Both rely on the server to receive and
process each and every request performed by either. As mentioned above, the interfaces define what is possible for the different users of the bookstore. Besides this, the return types, such as the {\tt ImmutableStockBook} prevents unintended use. The simple solution would simply be to return the StockBook, but then {\tt ``clever users''} could access the information only intended for server-side operations.

\subsubsection{}
Yes it is, as there is no way of communicating directly between the client and
the bookstore. All messages are passed through the locally instantiated server.
The locality has no effect on the outcome.

\subsection{Question 3}

\subsubsection{}
Indeed, there is. The naming service is the communication bridge between the
client and business, which is based on so-called message tags located in \\
{\tt com.acertainbookstore.server.BookStoreHTTPMessageHandler}. Each tag defines
a request, which is relayed by the server to the recipient, processed and then
a response is then given back.

\subsubsection{}
...

\subsection{Question 4}
In the architecture, this can be seen from the way HTTP communication is handled. Clients send and receive messages to the bookstore using the Jetty 8 HTML client/server library, wrapped in the {\tt SendAndRecv} method. Since the client does not attempt to resend messages, and since the bookstore servers plan for handling errors is simply to pass the on to the client, the architecture exhibits an {\it ``At most once''} behavior.

\subsection{Question 5}

\subsubsection{}
...

\subsubsection{}
...

\subsection{Question 6}

\subsubsection{}
While there can be multiple instances of clients, there is only one instance of
the server. So the bottleneck occurs in the server.

\subsubsection{}
...

\subsection{Question 7}

\subsubsection{}
Yes, as certain actions performed on a web proxy can be handled different in
such events.

\subsubsection{}
Definitely. For mere retrieval of objects, this could mask server-side
failures. However, in order to perform other actions, such as buying books,
may require more features of the underlaying API. Specifically, in such cases
there should be an order status present, that once an order was submitted would
be marked as pending --- that is, the server has yet to verify that such an
order can be processed.

\subsubsection{}
As pointed out in the previous question, it would require some notion of
yet-to-be-determined responses, and as such, a reflection of this should be
present in the book store API.


